\chapter{结束语}
\label{chap:conclusion}

本章对全文工作进行总结,并进一步展望未来工作。

\section{工作总结}
随着摩尔定律逐渐失效,人们开始寻找不同的思路以从不同的角度来继续提升芯片的性能。
堆叠技术和异构众核加速技术越来越受到业界的关注,也同时应用到了不少当代商业化的处理器中,并取得了不错的性能。
另一方面,随着云计算的不断发展和广泛应用,堆叠异构系统为云计算提供强大的计算能力。
然而,云计算的软硬件资源以服务的形式同时提供给大量的用户,这种多租户环境下如何优化系统的性能成为一个研究重点。
本文面向堆叠异构系统,通过应用透明的软硬件策略,解决内存超额配置、多任务切换管理以及互连网络的负载不均衡问题。
具体而言,本文的主要贡献包括:

(1)提出了一种内存超额配置管理框架

如今云服务提供商往往提供给租户相比于其硬件资源更多的资源,以提高其资源利用率。
而对于一些对资源需求要求高的用户时,很容易出现内存不足的问题。
虽然当代GPU由于统一虚拟内存和实时按需取页的功能出现能够支持内存的超额配置,但我们在真实GPU上的测试表明,应用程序在内存超额配置的情况下会遭遇严重的性能下降,某些情况甚至会导致宕机。
我们观察了不同的应用程序的访存特征,发现内存超额配置产生的原因也不一样。
因此我们提出了一种内存管理框架,根据应用程序的类型动态选择主动数据页逐出技术、内存感知的并行度控制技术和内存容量压缩技术的最佳组合来缓解内存超额配置导致的性能损失问题。

实验结果表明,对于规则应用程序我们几乎能够完全恢复内存超额配置带来的性能损失。
相比于当前的基准结构,我们的内存管理框架能够在内存超额配置的情况下,分别为数据共享的规则应用程序和非规则应用程序提供60.4\%和270\%的平均性能提升。

(2)提出了一种动态采用检查点备份的主动抢占机制

多任务处理机制已经在异构系统中被广泛应用。
而抢占机制则是多任务处理中必不可少的一环。
抢占机制能够满足应用程序的服务质量要求,为多任务切换提供更多选择。
然而,GPU由于其单指令多线程的特性,其上下文大小相比于CPU的上下文大小显著增加,上下文切换的开销也越来越大。
我们观察了GPU内核函数的启动过程,利用动态检查点备份的方法,实现了一种动态主动的抢占机制。

实验结果表明,我们的主动检查点备份的抢占机制平均降低了58.6\%的抢占延迟和23.3\%的上下文开销。
GPU的平均抢占延迟也可以被降低到3.6$\mu$s,应用程序可以更容易地满足服务质量的要求,更好地支持多任务机制。

(3)提出了一种动态延迟感知的2.5维堆叠片上网络负载均衡策略

2.5维堆叠片上网络是一种面向硅中介层堆叠的新型结构,利用硅中介层上的大量连线资源,新增加一层网络结构分担网络流量。
然而,当前的设计将上层网络用于计算核心节点之间的协议通信,而下层网络用于访存通信。
我们对于PARSEC测试集的测试发现,上下层网络会出现严重的负载不均衡。
为解决这个问题,我们提出了基于延迟感知的动态负载均衡策略。

实验结果表明,我们的负载均衡策略相比于基准设计、目的节点拥塞检测的策略和缓存感知的策略分别有45\%、14.9\%和6.5\%的性能提升。
我们的延迟感知的负载均衡策略相比于基准设计仅多消耗7.7\%的面积资源和5.8\%的功耗。



\section{未来工作}
本文紧紧围绕着``面向堆叠异构系统的应用透明策略''这个目标,采用应用透明的软硬件策略,解决堆叠异构系统中的内存容量瓶颈问题、多任务切换问题和网络负载不均衡问题。
我们提出的优化策略经实验,不仅取得了较大的性能提升,同时硬件开销非常低。
因此,本文的研究一方面在理论上为后续工作提供了指导,另一方面对于工程实践也有实际意义。
今后,我们将在以下几个方面进一步研究:

(1)内存超额配置管理框架

在内存超额配置的相关研究中,我们目前的研究仅停留在单GPU处理器阶段,且面向的应用程序也相对简单。
下一步工作将进一步面向多处理器的异构系统,考虑如何处理好异构加速器与CPU处理器之间的通信以及异构加速器之间的访存通信问题。
在内存容量不足时,如何在多个存储单元之间对用户透明地调整数据的位置并进行调度。
另一方面,面向当前的深度学习应用,由于其模型规模不断增大,其训练时要求的存储空间也越来越大,如何针对这种流行且具有特定规则的应用程序和框架进行优化也是我们下一步工作的重点。
解决这些问题有助于进一步克服``存储墙''问题,提升堆叠异构处理器的可伸缩性。

在工程实践方面,GPU和DSP等异构系统具有一定的相似性。GPU中的内存超额配置管理框架可进一步扩展到DSP等异构系统中。
结合特定异构系统的特点来实现这些策略的应用。


(2)主动抢占机制

在基于检查点备份的主动抢占机制研究中,当前的研究对于内核函数启动的整个流程研究还不够细致。
下一步工作将进一步在真机上探索内核函数启动过程的利用空间,寻求更加准确的预测和估计方法。
另一方面,在多任务切换的研究方面,我们依然需要考虑在上下文切换的过程中如何在高带宽利用率和低抢占开销中权衡。

在工程实践方面,目前商业化的GPU没有提供给用户抢占的控制接口,在这方面的深入研究有助于开发其工程价值。

(3)动态延迟感知的负载均衡策略

在2.5维堆叠片上网络的负载均衡研究中,我们当前的研究更偏重于研究核间通信和访存通信的负载不均衡性。
在当前商业化的异构系统中,各计算核心之间并没有通信,但多GPU之间会存在较多的通信。
因此,如何将延迟感知的负载均衡策略扩展到大规模多GPU系统非常具有挑战性。
解决这一问题,有助于为数据中心网络提供负载均衡解决方案,提升云计算性能。

在工程实践方面,负载均衡策略的应用主要包含两方面难点,一方面是收集拥塞信息部件的实现;另一方面是拥塞信息传输环网的实现。
需要进一步满足时序和硬件开销的要求。




