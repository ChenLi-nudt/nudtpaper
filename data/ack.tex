
%%% Local Variables:
%%% mode: latex
%%% TeX-master: "../main"
%%% End:

\begin{ack}
回顾这四年多的博士生涯,心中感慨万千。有喜悦、有痛苦、有挣扎、也有激动,但更多的是感激。

首先,衷心感谢我的导师郭阳研究员。自12年进入师门,这七年时间郭老师在学习、科研和生活等方面都一直给予悉心的指导和真诚的关怀,也使我度过了许多艰难的坎坷。
在我攻读博士学位期间,郭老师不但为我的研究指明了方向,营造了相对宽松的学术研究氛围,还创造了许多宝贵机会,使我能够集中专注于科研攻关,完成博士课题。
郭老师待人宽厚、治学严谨、视野宽广,这些品质都深深地影响了我,让我受益终身。

衷心感谢匹兹堡大学的杨峻教授和张有弢教授,在美国匹兹堡大学联合培养的两年时间里,两位教授让我接触到学术领域最前沿的研究思路和方法。
与杨老师在研究思路的确立、实验方案的设计以及论文思路上充分的讨论和细致的交流让我在学术研究上进步飞速,视野也更加开阔。
感谢卡内基梅隆大学的博士后Rachata Ausavarungnirum、德克萨斯大学奥斯丁分校的Christopher J. Rossbach教授和苏黎世联邦理工学院的Onur Mutlu教授。
与他们的合作使我与国际学术最前沿有了更深入的接触,他们的帮助使我的课题能够做得更加深入,对自己的标准和要求也变得更高。
这些经历都是非常难得和宝贵的。

衷心感谢马胜老师在我学术道路的起步过程中的引导和帮助,他严谨细致的科研风格和优秀的论文写作技巧都让我受益匪浅。
马老师与我分享了大量宝贵的经验,也一直是我学习的榜样。

感谢师门的张龙、刘畅、王子聪、袁珩洲、张军阳、蒋艳德等同学,我们在攻读博士学位的艰辛道路上同甘共苦、互相关心、互相支持、共同进步。
这都将是值得珍藏的经历。感谢师门其他的师弟师妹们,一起营造了Redsun良好的科研和学习氛围。
感谢在匹兹堡大学研究组里的刘基伟、张现伟、赵磊、温闻、王茹嘉、Andrew、Tyler、辛鑫、邓全、Yuyu、陈正国对我科研和在美生活上的帮助。
感谢Andrew对我英语口语提高的帮助。
感谢同在匹大访问的科大同学邓全、钱程、张江伟、陈正国、于齐,即使身在异国他乡,我们依然团结友爱、互相支持,营造了非常温馨的氛围。

感谢我的家人,谢谢他们的鼓励和支持,谢谢他们的默默奉献,是他们成就了我的学业。
感谢他们为我付出的一切,我将铭记于心,再多的文字也无法表达我的感激。

感谢我的爱人刘悦女士对我的陪伴和支持。虽然因为求学和工作,我们分隔两地,甚至分隔两国。
但她在生活上和精神上对我的帮助、支持和鼓励是我不断前进的强大动力。
未来,让我们无惧风雨,共同前行。

最后,向在百忙之中能够抽出时间对我的论文进行评审并提出宝贵意见的各位专家和教授致以诚挚的谢意!

\end{ack}
